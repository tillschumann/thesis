%\documentclass[german, twoside, parskip]{VRThesis} % dies ist eine deutsche Diplomarbeit
\documentclass[english,twoside, parskip]{VRThesis} % this is an english master thesis

% replace this by the title of the thesis
\title{This is the Title of my Thesis}
% replace this by the authors name
\author{Matthias Mustermann}
% replace this by the date of issue (Abgabedatum), for PhD thesis date of the exam
\date{\today}                 

% options for master thesis and Diplomarbeit
% Matrikelnummer, student ID
\studentid{999 666} 
\fieldofstudy{Informatik}
%\fieldofstudy{Software Systems Engineering}
\firstsupervisor{Prof. Dr. Torsten W. Kuhlen \\ Visual Computing Institute}
%\firstsupervisor{Prof. C. Bischof, Ph.D. \\ Institute for Scientific Computing}
\secondsupervisor{Univ.-Prof. Dr. A. Supervisor \\ Institute for Aplied Psychology}
\tutor{Dipl.-Inform. Hans Tutor}

% options for a PhD thesis
%\academictitle{eines Doktors der Naturwissenschaften}
%\academictitle{einer Doktorin der Naturwissenschaften}
%\bornin{Bonn}
%\firstsupervisor{Prof. C. Bischof, Ph.D.}
%\secondsupervisor{Univ.-Prof. Dr. A. Supervisor}
%\onlinesentence{}

\addtolength{\oddsidemargin}{+.2in}
\addtolength{\evensidemargin}{-.2in}

\begin{document} 

% generates the title page - use the proper one for your thesis
%\maketitle
\makecoverMaster % this is the thesis cover (a shorter version of the title page)
\maketitleMaster % this is the thesis title page
%\maketitlePhD
%\maketitlePhDfinal

% generates the statement (Erkl�rung) page for master thesis and Diplomarbeit
\makestatement

% generates the table of contents (Inhaltsverzeichnis)
\tableofcontents

\chapter{Einleitung}
\section{�berblick}
Dieser Text dient als Beispiel f�r das \LaTeX-Template der Virtual Reality Gruppe der RWTH Aachen. Er enth�lt au�erdem einen kurzen �berblick �ber spezielle Formatierungskonventionen.

Dir Vorgaben dieses Dokuments entsprechen weitgehend der KOMA-Script Klasse \texttt{scrbook}. Au�erdem wurden einige spezielle Anpassungen f�r die VR-Gruppe vorgenommen.

\section{Neue Befehle}
Um ein einheitliches Layout der Arbeiten zu gew�hrleisten, stellt die Dokumentklasse \texttt{VRThesis} nicht nur einen einheitlichen Satzspiegel bereit, sondern auch spezielle Befehle f�r einheitliche Titelseiten. 

\paragraph{Titel einer Diplomarbeit}
Speziell f�r Studenten gibt es den Befehl \texttt{maketitleMaster} f�r Diplomarbeiten (deutsch) und Masterarbeiten (englisch), der ein einheitliches Layout f�r Titelbl�tter von Diplom- und Masterarbeiten bereitstellt. Vorher m�ssen folgende felder gesetzt werden
\begin{itemize}
\item \texttt{title} Titel der Arbeit
\item \texttt{author} Name des Autors
\item \texttt{studentid} Matrikelnummer des Autors
\item \texttt{fieldofstudy} Studiengang, z.B. "`Informatik"' oder "`Software Systems Engineering"'
\item \texttt{firstsupervisor} Erstgutachter inkl. seines Institutes
\item \texttt{secondsupervisor} Zweitgutachter  inkl. seines Institutes
\item \texttt{tutor} Name des Betreuers; Ein zweiter Betreuer kann angegeben werden, wenn dieser durch ein "`$\backslash \backslash$ \&"' vom ersten getrennt wird.
\item \texttt{date} Tag der Abgabe
\end{itemize}

\paragraph{Titel f�r eine einzureichende Dissertationsschrift}
Die Promotionsordnung des FB 1 der RWTH-Aachen sieht spezielle Angaben vor, die auf dem Titelblatt einer Dissertation vorhanden sein m�ssen, wenn diese eingereicht wird. Aus diesen Informationen generiert der Befehl \texttt{maketitlePhD} automatisch ein einheitliches Titelblatt, das den Vorgaben entspricht. Der Befehl �bernimmt die folgenden Parameter (in der angegebenen Reihenfolge):
\begin{itemize}
\item \texttt{title} Titel der Arbeit
\item \texttt{author} Name des Autors inklusive (aktuellem) akademischem Grad und ggf. Geburtsnamen
\item \texttt{academictitle} Angabe des zu erlangenden Titels, z.B. "`eines Doktors der Naturwissenschaften"' oder "`einer Doktorin der Naturwissenschaften"'
\item \texttt{bornin} Geburtsort des Kandidaten, ggf. n�here geografische Lage des Geburtsortes
\end{itemize}

\paragraph{Titel der Pflichtexemplare einer Dissertationsschrift}
Die Pflichtexemplare sind nach einer bestandenen Doktorpr�fung abzugenben. Auch hier gibt es spezielle Vorgaben, die sich von denen der einzureichenden Version unterscheiden. Diese werden vom Kommando \texttt{maketitlePhDfinal} verarbeitet. Zus�tzlich zu den oben genannten Parametern m�ssen angegeben werden:
\begin{itemize}
\item \texttt{firstsupervisor} Erster Berichterstatter
\item \texttt{secondsupervisor} Zweiter Berichterstatter
\item \texttt{date} Datum der m�ndlichen Pr�fung
\end{itemize}
Wird die Dissertation elektronisch ver�ffentlicht, muss der Satz "`Diese Dissertation ist auf den Internetseiten der Hochschulbibliothek online verf�gbar."' hizugef�gt werden.

\chapter{Bemerkungen}

PLEASE, read the documents l2tabu.pdf (or l2tabuEN.pdf), short-math-guide.pdf and scientific-english.pdf before you start writing your thesis.

\section{Mathematische Formeln}
Eine abgesetzte Formel wird mittels der Umgebung \texttt{displaymath} eingef�gt. 
\begin{displaymath}
E=m\cdot c^2
\end{displaymath}
Die Verwendung von \$\$...\$\$ sollte vermieden werden, da es sich hierbei um einen "`Plain-TeX-Befehl"' handelt, der unter Anderem Absatzabst�nde nicht immer korrekt behandelt. 

Will man mehrere Formeln untereinander setzen, so nutzt man daf�r nicht, wie bisher, die \texttt{eqnarray}-Umgebung, sondern besser die Umgebung \texttt{align}.
\begin{align}
\frac{dx}{dt}&=v(x,t) \\
x(t_0) &= x_0
\end{align}


\section{Abbildungen}
Ein einfache Abbildung zeigt \ref{fig::Logo}. Hier ist darauf zu achten, dass die Zentrierung durch \texttt{\\centering} erreicht wird und nicht mittels der \texttt{center}-Umgebung.
\begin{figure}[h!]
\centering
\includegraphics[width=50mm]{rwth_vci_bild_schwarz_grau_cmyk}
\caption{\label{fig::Logo} Das Logo der Virtual Reality Group, RWTH Aachen University.}
\end{figure}


\end{document} 